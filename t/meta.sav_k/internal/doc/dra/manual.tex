\documentclass[12pt,final]{report}
\usepackage{fancyvrb}
\usepackage{float}
\usepackage{varioref}
\usepackage[nottoc]{tocbibind} % Doesn't play well with package "index" :-(
\usepackage{makeidx}
%\usepackage{index}  % prevents expansion in index entries, needed for progidx
                     % Makes hyperref for index go awry, so decided to use
                     % explicit protect instead: turned out not to be neeeded

%\proofmodetrue      %  <<<< (index) suppress for the final copy <<<<<<


\usepackage{parskip}[2001/04/09]
%\setlength{\parindent}{0pt}
%\setlength{\parskip}{2ex}

\usepackage[colorlinks]{hyperref}

\makeindex

%----------------------------------------------------------------------
%  MACROS

% Facilitate transition between article and book:
\newcommand*{\Chapter}[1]{\chapter{#1}}
\newcommand*{\Section}[1]{\section{#1}}
\newcommand*{\Subsection}[1]{\subsection{#1}}
%\newcommand*{\Chapter}[1]{\section{#1}}
%\newcommand*{\Section}[1]{\subsection{#1}}
%\newcommand*{\Subsection}[1]{\subsubsection{#1}}

% Three asterisks break the text up where even a subsection is not merited:
% use this as a separate paragraph:
\newcommand*{\Breakup}{\[\ast\ast\ast\]}

% Definition of a new concept:
\newcommand*{\Defconcept}[1]{\emph{#1}\index{#1}}
% Ditto if the word is different from the index entry:
\newcommand*{\Defconcepti}[2]{\emph{#1}\index{#2}}

% A short form of a tiny \marginpar:
\newcommand*{\mpar}[1]{\marginpar{\tiny#1}}

% A word in the text that should be indexed:
\newcommand*{\Index}[1]{#1\index{#1}}

% A word that is the object of discussion:
\newcommand*{\about}[1]{\emph{#1}}

% A small piece of a concrete program etc.:
\newcommand*{\prog}[1]{\texttt{#1}}

% A ``pattern'', e.g., a generic variable within a concrete call:
\newcommand*{\patt}[1]{\textit{#1}}

% A reference to an entity in a program:
\newcommand*{\progfrag}[1]{\about{#1}}

% An example of a term in the text:
\newcommand*{\term}[1]{\progfrag{#1}}

% A predicate specification:
\newcommand*{\pred}[1]{\about{#1}}

% Ditto within an index entry:
\newcommand*{\predidx}[1]{\about{#1}\index{#1@\pred{#1}}}

% Ditto without putting into the text:
\newcommand*{\predidxonly}[1]{\index{#1@\pred{#1}}}

% An index entry for a procedure:
\newcommand*{\progidx}[1]{\prog{#1}\index{#1@\prog{#1}}}

% Ditto without putting into the text:
\newcommand*{\progidxonly}[1]{\index{#1@\prog{#1}}}

% A reference to a figure:
\newcommand*{\Figref}[1]{Fig.~\vref{#1}}

% A reference to a chapter:
\newcommand*{\Chapref}[1]{chapter~\ref{#1}}

% A reference to a section:
\newcommand*{\Secref}[1]{Sec.~\protect\ref{#1}}

% A reference to a page:
\newcommand*{\Pageref}[1]{p.~\pageref{#1}}

% An indentation
\newcommand*{\ind}{\hbox{\hspace{2em}}}

% Eclipse:
\newcommand{\Eclipse}{ECL$^i$PS$^e$}

%Sicstus:
\newcommand{\Sicstus}{SICStus}



%----------------------------------------------------------------------
%  ENVIRONMENTS

% A warning:
\newenvironment{Warning}%
{\begin{quote}\textbf{Warning:}\itshape}%
{\end{quote}}

% Itemize with no topsep:
\newenvironment{Itemize}%
{\begin{list}{$\bullet$}%
    { \setlength{\topsep}{0pt}%
}}%
{\end{list}}

% A lightweight itemize:
\newenvironment{LightItemize}%
{\begin{list}{--}%
    { \setlength{\itemsep}{0.1ex}%
      \setlength{\topsep}{0pt}%
}}%
{\end{list}}

% Enumerate with no topsep:
\newcounter{Enumcnt}
\newenvironment{Enumerate}
               {\begin{list}{\arabic{Enumcnt}.}%
                   { \setlength{\topsep}{0pt}%
                     \usecounter{Enumcnt}%
               }}%
               {\end{list}}

% A lightweight enumerate, labeled with (a), {b):
\newcounter{lightenumcnt}
\newenvironment{LightEnumerate}
               {\begin{list}{(\alph{lightenumcnt})}%
                   {\setlength{\itemsep}{0pt}%
                     \setlength{\topsep}{0pt}%
                     \settowidth{\labelwidth}{(m)}%
                     \usecounter{lightenumcnt}%
               }}%
               {\end{list}}

%----------------------------------------------------------------------
\title{The DRA Interpreter\\
User Manual}

\author{Feliks Klu{\'z}niak\\
  \emph{Applied Logic, Programming Languages and Systems Lab}\\
  \emph{Department of Computer Science}\\
  \emph{University of Texas at Dallas}
}
\date{\small\today}

\bibliographystyle{plain}


%----------------------------------------------------------------------
\begin{document}

\maketitle

%%% reverse of title page
\thispagestyle{empty}
\setcounter{page}{0}

\fbox{\small
  \begin{minipage}{\textwidth}
    NOTICE:\\

    \copyright 2009 University of Texas at Dallas

    \mbox{}

    Developed at the Applied Logic, Programming Languages and Systems (ALPS)
    Laboratory at UTD by Feliks Klu{\'z}niak.

    \mbox{}

    Permission is granted to modify this text, and to distribute its original
    or modified contents for non-commercial purposes, on the condition that
    this notice is included in all copies in its original form.

    \mbox{}

    THE SOFTWARE IS PROVIDED ``AS IS'', WITHOUT WARRANTY OF ANY KIND, EXPRESS
    OR IMPLIED, INCLUDING BUT NOT LIMITED TO THE WARRANTIES OF
    MERCHANTABILITY, FITNESS FOR A PARTICULAR PURPOSE, TITLE AND
    NON-INFRINGEMENT. IN NO EVENT SHALL THE COPYRIGHT HOLDERS OR ANYONE
    DISTRIBUTING THE SOFTWARE BE LIABLE FOR ANY DAMAGES OR OTHER LIABILITY,
    WHETHER IN CONTRACT, TORT OR OTHERWISE, ARISING FROM, OUT OF OR IN
    CONNECTION WITH THE SOFTWARE OR THE USE OR OTHER DEALINGS IN THE
    SOFTWARE.
\end{minipage}
}

\vfill  %%%%%%%%%

{\footnotesize
  All comments, queries and suggestions about this manual or the software
  are welcome. The author's e-mail address is
  \prog{feliks.kluzniak@utdallas.edu}.}
%%% end reverse of title page

\tableofcontents

\section{Introduction}

Prolog, like most logic programming languages, offers backward
chaining as the only reasoning scheme.  It is well known that sound
and complete reasoning systems can be built using either exclusive
backward chaining or exclusive forward chaining \cite{Nilsson80}.
Thus, this is not a theoretical problem.  It is also well understood
how to ``implement'' forward reasoning using an exclusively backward
chaining system and vice versa.  Thus, this need not be a practical
problem.  In fact, many of the logic-based languages developed for AI
applications \cite{DUCK,MRS,Petrie88,Fritzson88a} allow one to build
systems with both forward and backward chaining rules.

There are, however, some interesting and important issues which need
to be addresses in order to provide the Prolog programmer with a
practical, efficient, and well integrated facility for forward
chaining.  This paper describes such a facility, \pfc\ , which we have
implemented in standard Prolog.

The \pfc\ system is a package that provides a forward reasoning
capability to be used together with conventional Prolog programs.  The
\pfc\ inference rules are Prolog terms which are asserted as facts
into the regular Prolog database.  For example, Figure
\ref{fig:pfcrules} shows a file of \pfc\ rules and facts which are
appropriate for the ubiquitous kinship domain.

\begin{figure}[bhp]
\figline
\small
\begin{verbatim}
spouse(X,Y) <=> spouse(Y,X).
spouse(X,Y),gender(X,G1),{otherGender(G1,G2)}
     =>gender(Y,G2).
gender(P,male) <=> male(P).
gender(P,female) <=> female(P).
parent(X,Y),female(X) <=> mother(X,Y).
parent(X,Y),parent(Y,Z) => grandparent(X,Z).
grandparent(X,Y),male(X) <=> grandfather(X,Y).
grandparent(X,Y),female(X) <=> grandmother(X,Y).
mother(Ma,Kid),parent(Kid,GrandKid)
      =>grandmother(Ma,GrandKid).
grandparent(X,Y),female(X) <=> grandmother(X,Y).
parent(X,Y),male(X) <=> father(X,Y).
mother(Ma,X),mother(Ma,Y),{X\==Y}
     =>sibling(X,Y).
\end{verbatim}
\caption[Pfc Rules]{{\bf Examples of \pfc\ rules which represent common kinship relations}}
\label{fig:pfcrules}
\figline
\end{figure}



The rest of this manual is structured as follows.  The next section
provides an informal introduction to the \pfc\ language.  Section
three describes the predicates through which the user calls \pfc\.
The final section gives several longer examples of the use of \pfc\

\subsection*{Getting and installing \pfc\ }

Look for \pfc\ on ftp.cs.umbc.edu in /pub/pfc/.


\Chapter{The interpreted programs\label{chap:programs}}



%-------------------------------------------------------------------------------
\Section{Limitations\label{sec:limitations}}

The interpreter does not support full Prolog.  Here are the main limitations
of the interpreted language:
\begin{Enumerate}

\item
  The interpreted program must not contain cuts\index{cut} (i.e., occurrences
  of \pred{!/2}\index{"!/2@\pred{"!/2}}).  Use of the conditional
  construct\index{conditional construct} is permitted, as is the use of
  \predidx{once/1}.

\item
  The interpreted program must not contain variable literals\index{variable
    literal}.  It may contain invocations of \predidx{call/1}, but if the
  argument of \pred{call/1} is not properly instantiated at runtime, you will
  get an error message and the interpreter will quit.\footnote{
    In some cases the interpreter can verify beforehand (i.e., at
    ``compile-time'') that the argument of an occurrence of \pred{call/1}
    cannot be instantiated at run-time, and it will then raise a fatal error.
    The check is quite conservative, so the absence of such an error message
    does not mean that the program is safe in that respect.}

\item
  The repertoire of built-in predicates\index{built-in predicates} recognized
  by the interpreter is somewhat limited.  This is done by design, mostly to
  facilitate porting to different Prolog systems.

  The recognized built-ins are declared in the file \prog{dra\_builtins.pl},
  and new declarations can be added as the need arises.  For most built-ins
  just adding another line to the file will suffice, but a few might
  require special treatment by the interpreter.\footnote{
    Having a file wherein you specify the names of built-in predicates you
    actually want to use does have its advantages.  Some logic programming
    systems (e.g., \Eclipse{}) support a very extensive set of libraries that
    define built-in predicates whose names are treated as reserved even if
    you don't use the libraries.  As a result, many names that you might
    reasonably want to use in your programs are not available to you.}
\end{Enumerate}

If these limitations seem too strict, you may in some cases get around them
by separating your program into two layers: see \Secref{sec:support}.


%-------------------------------------------------------------------------------
\Section{The notion of ``support''\label{sec:support}}

The interpreter provides you with an opportunity to divide your program into
two layers: an upper layer which makes use of the special facilities provided
by the interpreter (i.e., tabling and/or coinduction), and a lower layer of
``support'' software that requires only standard Prolog.  This can be useful
for increasing efficiency: the support layer will be compiled just as all
other ``normal'' Prolog programs.  An additional advantage is that the
support layer can use the full range of built-in predicates available in the
host logic programming system, and in particular the cut.

The interface between the two layers consists of a handful of entry-point
predicates, each of which is  declared by a directive similar to the
following one:\\
\ind\prog{:- support check\_consistency/1.}%
\label{dir:support}\progidxonly{support}\\
Please note that this directive cannot be entered interactively: it must be
included in the text of the upper layer part of your program.

The support declaration means that the metainterpreter should treat the
declared predicate as a built-in, i.e., just let Prolog execute it.

The support layer cannot invoke the upper layer, so there is no need to
declare those predicates in the support layer that are not directly invoked
by the upper layer.

Predicates that are declared as ``support'' (and those that are---directly or
indirectly---called by them) must be defined in other files.
To compile and load such a file, use the following directive in the text of
your program:\\
\ind\prog{:- load\_support(~\patt{filename}~).}%
\label{dir:load-support}\progidxonly{load\_support}\\
In this context, the default extension of the \patt{filename} will be the
default extension used by the host logic programming system for names
of files that contain Prolog code.%
\index{default extension}%
\index{extension of file name!default}%
\index{file!name!default extension}



%-------------------------------------------------------------------------------
\Section{Declaring ``entry points''\label{sec:entry-points}}

Before execution begins, the interpreted program is subjected to a number of
sanity checks.  One of these is a check whether every defined predicate is
actually called from somewhere (i.e., whether there is no dead code).

Since it is not unusual for a program to contain a handful of such predicates
on purpose (they are intended as ``entry points'' that are to be invoked from
a query),
the user can declare them by using a directive similar to the following:\\
\ind\prog{:-~top~p/1,~q/2.}\label{dir:top}\progidxonly{top}\\
The declaration is given only to suppress warnings.  However, it is an error
for an undefined predicate or a support predicate to be so declared.



%-------------------------------------------------------------------------------
\Section{Declaring dynamic predicates\label{sec:dynamic}}

To declare a predicate whose clauses are asserted and/or retracted by the
interpreted program, use\index{predicate!dynamic}\\
\ind\prog{:-~ dynamic~p/k.}\label{dir:dynamic}\progidxonly{dynamic}



%-------------------------------------------------------------------------------
\Section{Hooks\label{sec:hooks}}\index{hook}

The program may contain clauses that modify the definition of the
interpreter's predicate \predidx{essence\_hook/2} (the clauses will be
asserted at the front of the predicate, and will thus override the default
definition for some cases).  The interpreter's default definition is\\
\ind\prog{essence\_hook(~T,~T~).}

This predicate is invoked, in certain contexts, when:
\begin{LightItemize}
  \item
    two terms are about to be compared (either for equality or to check
    whether they are variants of each other);
  \item
    an answer is tabled;
  \item
    an answer is retrieved from the table.
\end{LightItemize}

The primary intended use is to allow suppression of arguments that carry only
administrative information and that may differ in two terms that are
considered to be ``semantically'' equal or variants of each other.

For example, the presence of\\
\ind\prog{essence\_hook(~p(~A,~B,~\_~),~~p(~A,~B~)~).}\\
will result in \prog{p(~a,~b,~c~)} and \prog{p(~a,~b,~d~)} being treated as
identical: each of them will be translated to \prog{p(~a,~b~)} before
comparison.

\begin{Warning}
This facility should be used with the utmost caution, as it may drastically
affect the semantics of the interpreted program in a fashion that could be
hard to understand for someone who is not familiar with the details of the
interpreter.
\end{Warning}

\Chapter{Running a program\label{chap:running}}


%-------------------------------------------------------------------------------
\Section{Loading the interpreter\label{sec:loading-dra}}%
\index{loading the interpreter}

The interpreter is written in Prolog.  It is distributed in source
form.%
%\footnote{
%  Please see the ``README'' files in the distribution tree: they will help you
%  find your way around.}

The interpreter is known to run on \Eclipse{}~6.0, \Sicstus{}~4.0 and
SWI~Prolog~5~7.  If you plan to run programs that take advantage of
coinductive programming, you might prefer to avoid \Eclipse{}, which has
somewhat inadequate support for cyclic terms.

The simplest way to proceed is to:
\begin{Enumerate}
\item
  start your logic programming system;
\item
  If you are using \Eclipse{} or \Sicstus{}, type in the following directive:\\
  \ind\prog{:-~[~'\patt{Path}/tabling/dra'~].}\\
  where \about{Path} is the path to the root of the distribution
  tree.
  If you are using SWI Prolog, the directive is:\footnote{You can use also\\
    \ind\prog{:-~[~'\patt{Path}/tabling/drapf'~].}\\
      if the interpreted program does not use cyclic terms (i.e., in
      particular if there are no coinductive predicates).  This version could
      be significantly faster, even faster than \Sicstus{}.
  }\\
  \ind\prog{:-~[~'\patt{Path}/tabling/drap'~].}
\end{Enumerate}

This will just load the interpreter, but you will still be interacting with
the host logic programming system.  \Secref{sec:loading-prog} describes how
to start the interpreter.

The interpreter is encapsulated in its own module, called \about{dra} (or
\about{drap} in the case of SWI Prolog).  So if
you are running \Eclipse{}, you will probably find it more convenient to import
the module by writing\\
\ind\prog{:- import dra.}\label{import-dra}\progidxonly{import}\\
immediately after loading the interpreter.

It may well be that things have been installed differently on your site.
This might be because the interpreter has been modified to run with a
different Prolog system, or because an immediately-loadable version has been
made available in some standard directory. The person responsible for the
local installation of the interpreter will provide you with more details.


%-------------------------------------------------------------------------------
\Section{Loading a program\label{sec:loading-prog}}%
\index{loading a program}

Once you have loaded the interpreter into your logic programing system, you
may want to load and run a program in the interpreter. This is done by
writing\\
\ind\prog{prog(~\patt{filename}~).}\progidxonly{prog}%
\footnote{
  If you are running in \Eclipse{}, and have not imported the module \about{dra}
  (as explained in \Secref{import-dra}), you must write \prog{dra:prog}
  instead of \prog{prog}.
}\\
\patt{filename} should be the name of the file that contains your program.
If the name is given with no extension, it will be automatically extended
with \prog{.tlp}.%
\index{default extension}%
\index{extension of file name!default}\index{file!name!default extension}
If the name should have a different extension, you must type in the entire
name, enclosed in single quotes, e.g.,\\
\ind\prog{prog(~'myfile.pl'~).}\\
Quotes must also be used if the file is not in the current directory and you
are providing an absolute or relative path.

As the file is being read and loaded, directives and queries are interpreted
on-the-fly. Each query is evaluated to give all solutions (i.e., as if the
user kept responding with a semicolon): to avoid that you can use the
built-in predicate \predidx{once/1} in the queries.

You should be aware that loading a program obliterates all traces of
previously loaded programs, including the contents of the answer table.  If
you are interested in re-running your program from scratch (so that it does
not take advantage of answers that were already tabled), you can just load it
again.



%-------------------------------------------------------------------------------
\Section{Interacting with a loaded program\label{sec:interacting}}

%%%
\Subsection{The interactive mode\label{sec:interactive-mode}}%
\index{interactive mode}

After the file is loaded (and all the directives and queries it contains are
executed), the interpreter enters interactive mode.  This is very much like
the usual top-level loop, except that it is the interpreter---and not the
underlying logic programming system---that evaluates queries and executes
directives.

In the interactive mode the interpreter will read your input and act on it.
Input consists of a term, terminated by a fullstop (i.e., the period
character) and immediately followed by a newline (i.e., you must press the
\prog{ENTER} key).\footnote{
  You cannot input more than one term per line. On \Sicstus{} all characters
  between the fullstop and the newline will be ignored.  On \Eclipse{}, if the
  answer to your query is ``\prog{Yes~~(more?)~}'', the remainder of the
  previous line will be treated as your input and the interpreter will seem
  to cease responding.  To get out of this state type in a fullstop followed
  by a newline.}
If a query succeeds, you will get a printout that looks like this:\\
\ind\prog{Yes~~(more?)~}\\
You should then type in a semicolon immediately followed by a newline (if you
want more answers), or just a newline (if you don't).

When you type in a term of the form ``\prog{:-~\patt{...}.}'', it will be
treated as a directive\index{directive}; when you type in a term of the form
``\prog{?-~\patt{...}.}'', it will be treated as a query\index{query}; when
you type in a term that does not begin with \prog{:-} or \prog{?-}, it will
also be treated as a query.

The difference between directives and queries is quite crucial, because the
names of the directives do not occupy the same name space\index{name space}
as the names of predicates.  If you type in, say,\\
\ind\prog{answers(~\_,~\_~).}\\
this will have nothing to do with the directive\\
\ind\prog{:-~answers(~\_,~\_~).}\\
and the interpreter will try to invoke the predicate \pred{answers/2} in your
program.  This may be a little confusing, but the good news is that you don't
have to worry about potential conflicts between the names in your program and
the names of the interpreter's directives.

Neither do you have to worry about conflicts between your program and the
interpreter itself. The interpreted program is loaded into a separate module
called \progidx{interpreted}.  If there is a support layer, it is loaded into
the module \progidx{support}.  I mention these names, because the host system
may show them in error messages if something goes horribly wrong.


%%%%
\Subsection{Resuming the interactive mode\label{sec:resuming-interactive}}%
\index{interactive mode!resuming}

To just enter interactive mode (without loading a new program)
invoke\footnote{
  Again, \prog{dra:top} in \Eclipse{}, if you have not imported\about{dra}.}\\
\ind\prog{top.}\progidxonly{top}

The interpreter does not allow you to input clauses directly from your
terminal, but it's good to have recourse to this call if you have exited
interactive mode (see below) or if the execution of the interpreter was
interrupted (either because of a fatal error, or because you pressed Ctrl-C
on your keyboard). The program that was most recently loaded is still there,
the answer table might have been populated, so you might want to resume
interactive mode.


%%%%
\Subsection{Exiting the interactive mode\label{sec:exiting-interactive}}%
\index{interactive mode!exiting}

To exit the interactive mode enter the end of file character
(\about{Ctrl-D}),%
\footnote{
  \about{Ctrl-D} appears not to work with tkeclipse.}
or just write\\
\ind\prog{quit.}\progidxonly{quit}


%%%%
\Subsection{Statistics\label{sec:statistics}}%
\index{statistics}

Just before the result of a query is reported, the interpreter produces a
printout with statistics\index{statistics} accumulated since the previous
such printout (or since the beginning, if this is the first printout during
the current session with the interpreted program). The printout looks like
this:\\
\ind\prog{[\patt{K}~steps,~\patt{M}~new~answers~tabled~(\patt{N}~in~all)]}\\
\patt{K},\patt{M} and \patt{N} are natural numbers. \patt{K} is the number of
evaluated goals, \patt{M} is the number of new additions to the answer table,
and \patt{N} is the current size of the answer table.

Please note that you might sometimes see new answers tabled in 0 steps: this
may happen when you ask for more results (by typing a semicolon) and the last
goal to be activated has still not completed its task.  You might also see
that new answers were added even though the final response is \prog{No}: this
only means that some auxiliary goals were successful, while the main one was
not.


%%%%
\Subsection{Print depth\label{sec:print-depth}}%
\index{print depth}

When a query succeeds, the instantiations of its variables should be printed
upto a certain maximum depth.  The default value in the distributed version
of the interpreter is 10.  The maximum depth can be changed from the
interpreted program (or interactively from the top-level) by invoking\\
\ind\prog{set\_print\_depth(~\patt{N}~)}\predidxonly{set\_print\_depth/1}\\
where \patt{N} is a positive integer.

Please note that with some Prolog implementations this might not prevent a
loop if the printed term is cyclic (as will often happen for coinductive
programs).

Note also that the foregoing does not apply to invocations of built-in
predicates in the interpreted program.  It is up to the user to apply the
built-in that is appropriate for the host logic programming system.  For
example, in the case of \Sicstus{}, use
\prog{write\_term(~T,~[~max\_depth(~10~)~]~)}, rather than just \prog{write(
  T )}, if you expect the instantiation of \prog{T} to be cyclic.



%-------------------------------------------------------------------------------
\Section{Including other files\label{sec:including}%
\index{including a file}\index{file!inclusion}}

To include files (interactively or from other files) you can use the usual
Prolog syntax:\\
\ind
\prog{:-~[~\patt{filename1},~\patt{filename2},~\patt{...}~].}%
\label{dir:include}\\
The default extension is \prog{.tlp}.%
\index{default extension}%
\index{extension of file name!default}\index{file!name!default extension}

Please note that including a file with \prog{:-~[~\patt{filename}~].}  and
loading a program with \prog{prog(~\patt{filename}~).} are very different
actions. When the interpreter includes a file, the contents are just
added to its memory. When it loads a program, it first (re)initializes
itself, wiping out the previously loaded program, all included files and the
answer table.



%-------------------------------------------------------------------------------
\Section{Inspecting the answer table\label{sec:answer-table}}%
\index{answer table}

In principle, the answer table is an auxiliary data structure that is, in
effect, accessed by normal queries.

However, the interpreter gives you the possibility of looking ``under the
hood'' by accessing the table directly.  This might be useful for assessing
the efficacy of your tabling declarations, or simply for satisfying your
curiosity.

To print out subsets of the current answer table, use\\
\ind\prog{:-~answers(~\patt{Goal},~\patt{Pattern}~).}%
\label{dir:answers}\progidxonly{answers}\\
where \patt{Goal} and \patt{Pattern} are terms.
This will print all those tabled answers that are associated with a variant
of the goal and unifiable with the pattern.  If the first argument is a
variable, the pattern will be used as a filter for all the answers in the
table.

To produce a dump of the entire table, just use\\
\ind\prog{:-~answers(~\_,~\_~).}



%-------------------------------------------------------------------------------
\Section{The ``wallpaper'' trace\label{sec:walpaper-trace}}

The interpreter does not incorporate an interactive debugger, but it can
produce a long trace of what happens during the execution of an interpreted
program.  This facility is useful mainly for helping to diagnose problems
with the interpreter: some of the information in the trace will not be easy
to understand for someone who does not know the details of the DRA
method~\cite{guo-gupta-dra}, and I will not try to explain it all here.
Still, you might sometimes be able to get some useful information from the
trace, e.g, about how new answers are added to the table.

To produce a wallpaper trace of what happens to some chosen predicates, use a
directive similar to the following:\\
\ind\prog{:-~trace~p/3,~q/0,~r/1.}\label{dir:trace}\progidxonly{trace}\\
If you want to trace all predicates, use\\
\ind\prog{:-~trace~all.}\\
These directives are cumulative.



\chapter*{Summary of directives\label{directives}}%
\addcontentsline{toc}{chapter}{Summary of directives}


{\small
An argument specified as \patt{PredSpec} can take three forms:
\begin{LightEnumerate}
\item
  A predicate specification written as \pred{name/arity}: for example
  \prog{foo/3} (in the short descriptions below we will assume this is the form
  that is used);
\item
  A sequence of such specifications, separated by commas: for example\\
  \prog{p/2,~q/1,~r/3};
\item
  The word \prog{all}, which specifies all predicates. (This cannot be used
  for \prog{support} and \prog{dynamic}!)
\end{LightEnumerate}
If the same kind of directive occurs a number of times, specifying different
predicates, the results are cumulative.  In particular, \prog{all} subsumes all
other predicate specifications.

\newlength{\DescWidth}
\setlength{\DescWidth}{16em}
\begin{tabular}{llr}
\emph{Directive:}      & \emph{Short description:}   \\

\prog{:-~[~\patt{filename}~].}
                   & \parbox[t]{\DescWidth}{
                        load a part of the program (\Pageref{dir:include})}\\

\prog{:-~answers(~\patt{Goal},~\patt{Pattern}~).}
                  & \parbox[t]{\DescWidth}{
                       inspect the answer table (\Pageref{dir:answers})}\\

\prog{:-~coinductive~\patt{PredSpec}.}
                   & \parbox[t]{\DescWidth}{
                       predicate is coinductive (old style)
                       (\Pageref{dir:coinductive})}\\

\prog{:-~coinductive1~\patt{PredSpec}.}
                   & \parbox[t]{\DescWidth}{
                       predicate is coinductive (new style)
                       (\Pageref{dir:coinductive1})}\\

\prog{:-~dynamic~\patt{PredSpec}.}
                  & \parbox[t]{\DescWidth}{
                        predicate is dynamic (\Pageref{dir:dynamic})}\\

\prog{:- load\_support(~\patt{filename}~).}
                   & \parbox[t]{\DescWidth}{
                      load (a part of) the support layer
                                                (\Pageref{dir:load-support})}\\

\prog{:-~old\_first.} &  \parbox[t]{\DescWidth}{
                                 change the order in which results are produced
                                 (\Pageref{dir:old-first})
                                 }\\

\prog{:- support~\patt{PredSpec}}
                   & \parbox[t]{\DescWidth}{
                       predicate is an entry point to the support layer
                                                   (\Pageref{dir:support})}\\

\prog{:-~tabled~\patt{PredSpec}.}
                   & \parbox[t]{\DescWidth}{
                            predicate is tabled (\Pageref{dir:tabled})}\\

\prog{:-~top~\patt{PredSpec}}
                   & \parbox[t]{\DescWidth}{
                        predicate is an entry point (\Pageref{dir:top})}\\

\prog{:-~trace~\patt{PredSpec}.}
                  & \parbox[t]{\DescWidth}{
                        trace the predicate (\Pageref{dir:trace})}
\end{tabular}
} % small


\newpage
\bibliography{bibliography}

\newpage
\printindex
\end{document}
